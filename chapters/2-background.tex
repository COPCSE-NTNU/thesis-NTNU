\chapter{Background}

\section{Automatic Identification Systems (AIS)}

\begin{figure}[htbp]  % order of priority: h here, t top, b bottom, p page
    \centering
    \includegraphics[width=1.0\textwidth]{figures/ais_positions}
    \caption{Vessel positions derived from 200 million AIS positional reports}
    \label{fig:ais_positions}
\end{figure}

As already mentioned in \cref{sec:topics_covered}, \acrfull{ais} was initiated by \acrfull{imo} and since 2004 every commercial and passenger vessel exceeding 299 \acrfull{gt} is required to carry an \acrshort{ais} transmitter. These transmitters broadcasts \acrshort{ais} messages following the \gls{aivdm} protocol. The \gls{aivdm} protocol contains two main types of reports: positional and static. The positional reports contains automatically collected information such as the transmitting vessel's \acrfull{mmsi} number, the current timestamp, and the vessel's current navigational data including the current geographical coordinates, \acrfull{sog}, \acrfull{cog}, true heading, \acrfull{rot}, and more. The static reports contain additional information about the vessel and its current voyage, some of which are manually inputted, such as the vessel's \acrshort{imo} number, name, dimensions, draft, intended destination and \acrfull{eta}. As an example, \cref{fig:ais_positions} shows a visualization of 200 million \acrshort{ais} randomly chosen positional reports from a collection of historical \acrshort{ais} positions for global collection of shipping vessels.

Regarding vessel identification there are mainly two values that are unique to a given vessel: the \acrshort{mmsi} and \acrshort{imo} numbers. Either of these should be unique on their own for a given vessel, however, \acrshort{mmsi} numbers can be recycled under certain conditions such as when a vessel is put out of commission while the \acrshort{imo} number is specific to a vessel's hull. Therefore, \acrshort{imo} is the preferred identifier, however, since the \gls{aivdm} protocol divides these identifiers in positional and static reports, both needs to be considered in order to use both static and positional \acrshort{ais} information.

\section{Additional vessel information and segmentation}

The collaborative company \acrfull{mo} has implemented a system for categorising vessels into different segments, subsegments, and further variations. This segmentations are based on various factors such as the dimensional data provided by \acrshort{ais} messages as well as details provided by external vessel information sources and even user and manual input. This segmentation of vessels is highly relevant to voyage patterns as vessels of different type and size travel to different ports and countries for different shipping companies. This is further shown in \cref{fig:segment_map} which shows, from an image of \acrshort{mo}'s web platform, how different subsegments of the dry bulk cargo segment travels in different areas of the world. Since this categorisation provides valuable insights into voyage patterns, vessel segmentation values are included in this thesis' proposed approach to vessel destination prediction.

\begin{figure}[htbp]  % order of priority: h here, t top, b bottom, p page
    \centering
    \includegraphics[width=1.0\textwidth]{figures/segment_map}
    \caption{\acrfull{mo}’s segmentation of vessels where yellow vessels are smaller than reds}
    \label{fig:segment_map}
\end{figure}

Moreover, \acrshort{mo} also has extensive vessel details for every vessel structured as a questionnaire in their product. This vast questionnaire contains information and fields from a combination of standards in the industry such as Q88\footnote{\url{https://corp.q88.com/}}. Data for the questionnaire is also collected from a number of external sources such as IHS Merkit\footnote{\url{https://ihsmarkit.com/index.html}} and DNV\footnote{\url{https://www.dnv.com/}}. Users of \acrshort{mo} also have the possibilty of suggesting changes to a public version of this questionnaire for any vessel. These changes are verified by \acrshort{mo} and published if the information proves accurate. This detailed description of vessels provides creates big potential for data analysis and a potential \acrshort{ml} model that is highly aware of specific vessels which ultimately may affect its traveling patterns.


\section{Trajectory similarity}

As will be further elaborated on in \cref{chap:related_work}, the current literature related to vessel destination predictions almost exclusively rely on some form of trajectory similarity. Therefore, trajectory similarity is also included in this thesis' proposed approach to vessel destination prediction. There are three main categories of trajectory similarity measurements: spatial, temporal and tempo-spatial. Regarding vessel trajectories derived from \acrshort{ais}, they are not likely to share similar time intervals values as vessels travel at different speeds and at different times, therefore, for the purpose of this thesis, only spatial trajectory similarity measures are considered. This assumption is further corroborated by \cite{ZHANG2020102729} that arrived at a similar conclusion in their work developing a \acrfull{ml} -based approach to trajectory similarity measurements.

There are a number of spatial trajectory comparison methods that have been widely used for different purposes. The most relevant are the Hausdorff distance \parencite{magdy2015}, Fréchet distance \parencite{magdy2015}, and \acrfull{sspd} \parencite{besse2015review}. Out of these, the \acrshort{sspd} method is the most appropriate as it handles trajectories of different shapes and lengths well which is beneficial when comparing a trajectory from an ongoing vessel voyage to a set of complete historical ones. \cref{fig:sspd} shows an example from \cite{besse2015review} where two trajectories are compared and their symmetric distances are calculated.

\begin{figure}[htbp]  % order of priority: h here, t top, b bottom, p page
    \centering
    \includegraphics[width=0.5\textwidth]{figures/sspd}
    \caption{SPD in the SSPD process of comparing two different trajectories \parencite{besse2015review}}
    \label{fig:sspd}
\end{figure}

The methods mentioned thus far are all algorithmic approaches to measuring similarities between trajectories, however, there are also \acrshort{ml}-based methods as well such as the approach proposed in \cite{ZHANG2020102729} that also compares their results to the aforementioned methods. They used a \acrfull{rf} to measure similarities between a traveling trajectory and every historical trajectory traveling from the same departure port in order to predict the next destination port for a vessel. They achieved a higher general accuracy when compared to algorithmic methods such as \acrshort{sspd}. Moreover, for similar purposes, some unsupervised clustering methods have also been applied to similar problems such as the \acrfull{dbscan} which is capable of sequentially finding patterns in points and trajectories. This approach is more frequently used in trajectory predictions on a small geographical extent such as for collision detection and anomaly detection.

\section{Machine learning (ML)}

\acrfull{ml} is an umbrella term describing...
