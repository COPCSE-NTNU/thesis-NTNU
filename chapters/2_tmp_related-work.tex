\chapter{Related work}

(Copied from RPP)

As mentioned mentioned in \cref{section:justifications_motivations_benefits}, historical AIS analysis has been an explored topic, especially within predicting vessels’ future destinations in short time intervals at very high geographical accuracy. The purpose of such research is mostly used for anomaly detection and collision avoidance where it is important to know a vessel’s coordinates for minutes up til an hour into the future. However, some of these methods seem to show some potential to be combined in order to predict vessel destinations further into the future, therefore, this research area has been included in relevant work as a separate category. The main category, on the other hand, includes methods that are directly applicable to longer range destinations or general availability, although there is much less research available within this area. The resulting two categories defined were the following:

\begin{enumerate}
    \item Vessel destination (or availability) prediction based on historical AIS data
    \item Short term navigational prediction based on historical AIS data
\end{enumerate}

\textit{Category 1} includes papers that used prediction methods to forecast either vessel availability, or supply, or methods applied to predict a vessel’s future destination port or region. Category 2 includes papers that used prediction or analytical methods to predict a vessel’s future positions on a smaller geographical scale or time-interval for different purposes than port destination prediction.

Finally, it was also considered that some prediction methods applied to availability could be non-AIS based such as considering patterns in cargo demand or other external economical factors. As there are many such factors that could have a varied amount of impact on availability, it was not feasible to thoroughly analyze these areas in this preliminary research of related work. For example, some factors might seemingly be completely unrelated to maritime shipping but have some manor of indirect impact on vessel traffic. Such cases would be hard to detect as the search area would be too large. Furthermore, existing literature in these areas does not mention external factors as having any impact on their prediction model. Therefore, this could be considered a limitation of existing work and could be interesting to be explored as future work of the Master’s thesis.

\section{Literature review}
\label{sec:lit_review}

To find papers relevant to \textit{category 1}, search queries were run on several well-established search engines. As an example, using the search engine provided by ScienceDirect\footnotemark had the following results:

\begin{itemize}
    \item `vessel destination' OR `ship destination' OR `vessel availability' resulted in \textbf{389} papers
    \item ais AND (`vessel destination' OR `vessel availability') resulted in \textbf{25} papers.
    \item ais AND (predicting OR forecasting) AND (`vessel destination' OR `vessel availability' OR `ship supply') resulted in \textbf{18} papers out of which only one paper proved relevant within this category.
\end{itemize}

From the resulting papers, only two papers were relevant to a degree that was satisfactory to answer the research questions.

To find papers relevant to \textit{category 2}, similar queries were run, however, instead of using the terms destination and availability, the term trajectory was used. As an example, using the search engine provided by ScienceDirect\footnotemark[\value{footnote}] had the following results:

\footnotetext{\url{https://sciencedirect.com}}

\begin{itemize}
    \item `vessel trajectory' OR ´ship trajectory' resulted in \textbf{421} papers
    \item ais AND (`vessel trajectory' OR `ship trajectory') resulted in \textbf{150} papers
    \item (ais OR `tracking data') AND (prediction OR estimation) AND (`vessel trajectory' OR `shipping route') resulted in \textbf{150} papers
    \item ais AND (predicting OR predict) AND (`vessel trajectory' OR `ship trajectory' OR `marine traffic') resulted in \textbf{132} papers
    \item ais AND (prediction OR predicting) AND (`vessel trajectory' OR `ship trajectory') resulted in \textbf{93} papers
\end{itemize}

From the resulting papers, several papers were relevant to a degree that was satisfactory to answer one or more of the research questions. Three of the most relevant papers from this category are presented in the following chapter.

\section{Answering the research questions}

As there was an abundance of research available within \textit{category 2}, three papers were selected for comparison as they represent three different relevant approaches. Behind the last two papers listed in \cref{tab:relevant_papers} especially, there were several more papers available describing similar approaches and results. These were selected to answer the research questions because they showed promise in the field and were also referred to by other papers as representative studies of research areas similar to \textit{category 2}.

\begin{table}
    \centering
    \begin{tabular}{p{1in} p{2.5in} p{1in}}
        \toprule
        \textbf{Paper} & \textbf{Query} & \textbf{Search Engine} \\
        \midrule
        \cite{lechtenberg2019} & ais AND (predicting OR forecasting) AND (`vessel destination' OR `vessel availability' OR `ship supply') & Google Scholar \\
        \midrule
        \cite{ZHANG2020102729} & ais AND (predicting OR forecasting) AND (`vessel destination' OR `vessel availability' OR `ship supply") & Google Scholar, ScienceDirect \\
        \midrule
        \cite{Pelizzari2016GeneticAF} & (ais OR "tracking data") AND (prediction OR estimation) AND ("vessel trajectory" OR "shipping route") & Google Scholar \\
        \midrule
        \cite{pallotta} & ais AND (prediction OR predicting) AND ("vessel trajectory" OR "ship trajectory" OR route) & Google Scholar \\
        \midrule
        \cite{Daranda2016NeuralNA} & ais AND (predicting OR predict) AND ("vessel trajectory" OR "ship trajectory" OR "marine traffic") & Google Scholar, Oria \\
        \bottomrule
    \end{tabular}
    \caption{Most relevant papers for further analysis}
    \label{tab:relevant_papers}
\end{table}

\paragraphheader{RQ 1: What prediction methods can be used to predict vessel availability?}

This question is applicable for \textit{category 1}, and as mentioned in \cref{sec:lit_review}, only a couple of papers were relevant, however, only one of these were directly applied to forecast vessel availability, the other was applied to predict destination:

\cite{lechtenberg2019} used a combination of several prediction methods in order to predict vessels’ next destination region, estimated time of arrival (ETA), and anchor time (AT) within regions. For predicting the next destination region the \textit{Markov Decision Process} was used as there are a limited number of possible regions (44) that a vessel can travel to. For the ETA and AT predictions, an \textit{XGBoost} method was applied. However, the extent of the regions was not disclosed, and although it was explained that port frequencies were used to determine regional availability, the accuracy of port frequencies was also not disclosed.

\paragraphheader{RQ 2: What prediction methods can be used to predict vessel destinations?}

\cite{ZHANG2020102729} was the second paper found which fitted within \textit{category 1}, and used a random forest approach to compare a given vessel’s current trajectory with all historical trajectories from the same departure port. They also used port frequency to normalize the results. In this way, they managed to achieve good results by combining both methods for predicting port destinations as well as city destinations. This method was unique as it considers multiple aspects of vessel voyages compared to the other methods.\\

The following papers were found within \textit{category 2}.\\

\cite{pallotta} developed their own Traffic Route Extraction and Anomaly Detection (TREAD) method which used DBSCAN to cluster AIS records transmitted within a predefined bounding area. Based on the most frequently traveled trajectories, the TREAD method could be applied to predict a current vessel’s future destination within a given bounding area by choosing the most frequently traveled path. Although, the main purpose of the TREAD method was not to predict vessel destinations but to provide a general framework for pattern analysis. The prediction aspect of the model was therefore not applied on a larger scale and thus no prediction performance metrics were disclosed.

\cite{Pelizzari2016GeneticAF} used a genetic algorithm to find the most optimal route between two ocean regions. The genetic algorithm trains on a dataset consisting of historical trajectories between two ports. It uses a generation-based (or evolutionary) training method and after circa 80 generations of training, it managed to find a good route between two ports. This route is not one of the historical trajectories, but rather its own unique “most optimal” route. This method could be applied to a global set of ports in order to establish a network of routes which could be used for prediction methods.


\cite{Daranda2016NeuralNA} also used the DBSCAN algorithm to cluster AIS records and combined it with an artificial neural network (ANN) which was used to predict turning points. So, DBSCAN was used to cluster records into routes, and the ANN was used to predict what routes the vessel would take next. The paper described accurate results within a 24 hour time interval but provides no specific accuracy metrics generalized on a large number of vessels or ports.

There were more papers relevant within \textit{category 2}, however, these three articles represent unique approaches and the most promising results while still being relevant. More methods were found within this category, but these were similar in approach and equally or less applicable. Therefore they did not provide any further insight into the research questions.


\paragraphheader{RQ 2A: What type of data did they rely on?}

All of the aforementioned methods relied on historical AIS data collected by a combination of satellite and land-based base stations. \cite{Pelizzari2016GeneticAF} also mentions using Long-Range Identification and Tracking (LRIT) data in addition to AIS.

\paragraphheader{RQ 2B: How much depth of the data was relevant to the results?}

\cite{Daranda2016NeuralNA}, \cite{Pelizzari2016GeneticAF}, \cite{pallotta}, and \cite{ZHANG2020102729} all exclusively relied on the navigational data supplied by the AIS protocol. The navigational part of the AIS protocol includes coordinates, speed over ground (SOG), rate of turn (ROT), course over ground (COG), and more. Furthermore, \cite{ZHANG2020102729} also considered port frequencies, however, this frequency is deducted from the navigational part of the AIS data. Similarly, \cite{lechtenberg2019} also used port frequencies to predict regional availability, ETA, and AT. As already mentioned in \cref{sec:problem_desc}, destination and ETA values are included in the AIS protocol, however, as they are manually inputted by crew members they are not accurate. This is reflected in the existing literature as none of the aforementioned methods takes these values into consideration. Thus, it seems that all the relevant research ultimately only considers the navigational part of the AIS protocol for future predictions.

\paragraphheader{RQ 2C: How successful were they at predicting vessel destinations?}

\cite{lechtenberg2019} claims a \textit{98\%} accuracy when it comes to predicting a vessel’s next region, however, this value is presumed to vary depending on the size of the regions which is not disclosed in the paper. \cite{ZHANG2020102729} claims to have achieved a \textit{66.57\%} accuracy level for port-based predictions and \textit{81.65\%} accuracy for city-based predictions. \cite{pallotta} does not disclose a general level of accuracy for their model which varies a lot depending on the amount of AIS records available within the predefined bounding area as well as how much repeatable traffic flows through it. For example, the paper mentions that \textit{95\%} of all messages were usable around the Strait of Gibraltar, but in the Indian ocean it was closer to \textit{40\%}. It is a similar story for \cite{Daranda2016NeuralNA} which shows an example of a very accurate route over a 24-hour interval but does not disclose a general level of accuracy for any given vessel at any given location. For \cite{Pelizzari2016GeneticAF}, accuracy is not a sensible metric because the method was never applied to predict a vessel’s future destination but rather finds the best routes. However, the paper mentions route predictions as a possible application, but this would entail training the model to find the best paths between every port in the world and compare them with current trajectories. However, the routes in the paper were evaluated to have a high value by experts in the field.

As there is very little research that directly concerns predicting port or region destinations it is hard to establish a general accuracy of the state of the art. The closest assumption would be around \textit{70\%} for port predictions and \textit{98\%} for region predictions.

\paragraphheader{RQ 2D: How applicable were they toward predicting vessel availability?}

\cite{lechtenberg2019} was directly applicable and applied toward predicting vessel availability with a global set of regions. \cite{ZHANG2020102729} was not directly applied to forecast availability although it shows a very promising and generic method of predicting vessel’s future destinations unrestricted from time intervals or geographical areas. It is therefore very applicable toward predicting availability if applied to a global set of vessels. The rest of the aforementioned papers does not seem applicable toward availability if not combined with other methods that enable them to give accurate predictions globally.

\paragraphheader{RQ 3: How can the quality of the prediction methods be ensured?}

\cite{lechtenberg2019} does not go very in-depth on this topic, however, the paper mentions dividing the data into \textit{90\%} training data and \textit{10\%} test data. The accuracy metric is taken from how much of the test data was accurate. \cite{ZHANG2020102729} did an extensive evaluation of their method including the five-folder cross-validation method to ensure the model is not over-fitted. This process includes dividing the data into five “folders” and using one folder at a time for evaluation and the others for training. If the general accuracy does not vary much across the evaluation folders, the model is not overfitted. \cite{pallotta} and \cite{Daranda2016NeuralNA} does not go into depth about how their models were evaluated, while \cite{Pelizzari2016GeneticAF} mentions the results were evaluated by experts in the field. None of the aforementioned papers include valuable insight into their data quality which would be vital to quality assurance. It is therefore considered a limitation of existing work and will be an important aspect to take into consideration in the Master’s thesis.

\paragraphheader{RQ 3A: What metrics, or measurements, are used to establish quality?}

It seems that accuracy is the main measurement used to establish quality. Furthermore, \cite{lechtenberg2019} mentions using both the “mean absolute error” (MAE) and the “root mean square error” (RMSE) as quality indicators. \cite{ZHANG2020102729} mainly uses “average prediction distance error” (APDE) as their quality indicator which is based on the distance between the predicted trajectory and the actual trajectory.

\paragraphheader{RQ 4: How extensive is the impact of considering segmentation for prediction methods?}

None of the papers found within the research areas considers any type of segmentation of vessels. It is, therefore, impossible to answer this research question based on the current state of the art of the problem area.