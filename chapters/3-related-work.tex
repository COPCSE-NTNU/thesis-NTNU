\chapter{Related work}
\label{chap:related_work}

The topic of \acrfull{ais} -based predictions has already been explored quite extensively, especially in recent years as \acrshort{ais} systems have become an enforced standard for commercial vessels in the industry. However, the \acrshort{ais} standard has mainly been applied for the purpose of maritime safety and navigation, and the existing academic work on this topic reflects this. Most of the related work consists of vessel trajectory predictions for the purpose of foreseeing possible future collisions or for detecting anomalies from established shipping lanes. These types of predictions are applicable for predicting a vessel's future position in a short time interval, in a limited geographical area, but with high positional accuracy. In order to establish the current state of the art of the topic area and establish to what extent the literature answers the proposed research questions, a literature review was conducted which is described in the following section.

\section{Literature review}
\label{sec:lit_review}

As indicated in Chapter/Section x, based on initial research into the thesis' topic area, there appears to be a trend toward a focus on short-term predictions for safety or navigational purposes. In contrast, this thesis aims at using \acrshort{ais}, and other attributes, for longer-term predictions, or more precisely, port destination predictions. However, because of the exploratory nature of the thesis, the literature review conducted was broad in order to include work that might have taken a different approach to solving the same problem. In order to organize the resulting papers, a categorical separation of papers based on motivation was defined as follows:

\begin{enumerate}
\setcounter{enumi}{-1}
    \item The paper's motivation deems it completely irrelevant to the topic area.
    \item The paper's motivation includes vessel predictions, but on a smaller time or geographical scale making it irrelevant for comparison.
    \item The paper's motivation includes destination predictions making it relevant for further analysis.
\end{enumerate}

\textit{Category 0} is defined to filter out papers that were irrelevant but could not be excluded by narrowing the search query. \textit{Category 1} includes papers that relate to the established trend mentioned earlier where the proposed method seems relevant on a small scale but is ultimately not applicable to the thesis' problem area. It also includes papers that consider relevant topics but not relevant solutions. Finally, the papers labeled with relevancy \texttt{2} falls within \textit{Category 2} and includes papers that fall within the same topic area and are relevant in terms of insight into the proposed research questions. In order to determine what papers fitted \textit{category 1} and \textit{2}, papers with a relevance higher than zero were further analyzed in order to determine the following attributes:

\begin{itemize}
    \item \textbf{Motivation} - what was the problem the paper aimed to solve.
    \item \textbf{Objective} - what was the proposed prediction method's objective (e.g. classification, prediction, clustering).
    \item \textbf{Data source} - what data source was used in the proposed solution.
    \item \textbf{Prediction method} - what prediction method was applied to solve the problem.
    \item \textbf{Geographical extent} - to what geographical extent was the solution applicable.
    \item \textbf{Time interval} - what were the limitations on prediction methods in terms of time or trajectory duration.
    \item \textbf{Validation method} - what methods were applied to validate the resulting prediction method.
    \item \textbf{Validation metrics} - what metrics were used when establishing the validity of the solution.
\end{itemize}

In this literature review, the following search engines/libraries where used to collect relevant research:

\begin{itemize}
    \item \textit{Scopus}\footnote{\url{https://www.scopus.com/}}
    \item \textit{Oria}\footnote{\url{http://ntnu.oria.no/}}
    \item \textit{ACM Digital Library}\footnote{\url{https://dl.acm.org/}}
\end{itemize}

These were chosen based on running test queries and evaluating the relevancy and range of the resulting papers. Furthermore, the proposed query had too many boolean operators for search engines like \textit{ScienceDirect}\footnote{\url{https://sciencedirect.com}}.

\subsection{Search query and filters}

The objective of the literature review was to conduct a broad search detecting papers related to multiple relevant topics such as \textit{vessel destination prediction}, \textit{vessel trajectory prediction}, \textit{vessel availability forecasting}, and \textit{maritime logistics}. Therefore, the search query used in the literature review was designed to find papers within multiple topics and was derived from testing multiple queries on multiple search engines. For instance, the following queries were tested using the search engine provided by \textit{ScienceDirect}:

\begin{itemize}
    \item ``vessel trajectory'' OR ``ship trajectory'' resulted in \textbf{421} papers
    \item ais AND (``vessel trajectory'' OR ``ship trajectory'') resulted in \textbf{150} papers
    \item ais AND (prediction OR predicting) AND (``vessel trajectory'' OR ``ship trajectory'') resulted in \textbf{108} papers
\end{itemize}

The above queries returned a large number of papers relevant to \textit{category 1}, so in order to find more relevant papers, more specific queries were also tested:

\begin{itemize}
    \item ``vessel destination'' OR ``ship destination'' OR ``vessel availability'' resulted in \textbf{389} papers
    \item ais AND (``vessel destination'' OR ``vessel availability'') resulted in \textbf{25} papers.
    \item ais AND (predicting OR forecasting) AND (``vessel destination'' OR ``vessel availability'' OR ``ship supply'') resulted in \textbf{18} papers.
\end{itemize}

Lastly, in order to find detect research approaching the same problem from a different direction such as not using \acrshort{ais} data, the following queries were also tested using \textit{Scopus} because of boolean operator limits on \textit{ScienceDirect}:

\begin{itemize}
    \item (vessel OR ship OR maritime) AND (destination OR availability OR supply) AND (prediction OR predicting OR forecasting OR logistics) resulted in \textbf{894} on \textit{Scopus}
    \item (vessel OR ship OR ``maritime logistics'') AND (destination OR availability OR supply) AND (prediction OR forecasting) resulted in \textbf{314} on \textit{Scopus}
    \item (vessel OR ship OR ``maritime logistics'') AND (destination OR availability) AND (predicting OR forecasting) resulted in \textbf{92} on \textit{Scopus}
\end{itemize}

The search terms that seemed to return the most relevant papers were combined into the final query used in the literature review shown in \cref{lst:search_query}.

\begin{lstlisting}[
    caption={Search query used in literature review},
    label=lst:search_query
]
ais AND (
    predict OR predicting OR forecast OR forecasting
) AND (
    vessel OR ship OR maritime
) AND (
    destination OR availability OR supply OR trajectory OR logistics
)
\end{lstlisting}

Moreover, the following filters were used to limit the search result:

\begin{itemize}
    \item The paper must be published in the last five years.
    \item The paper must be available in English.
    \item The paper must be available with the publisher subscriptions provided by \acrshort{ntnu}.
\end{itemize}

The search query was first run on the search engine \textit{Scopus}, thus the search query and filters were modified to the search engine's format as shown in \cref{lst:search_query_scopus}.

\begin{lstlisting}[
    caption={Search query used in Scopus including filters},
    label=lst:search_query_scopus
]
TITLE-ABS-KEY (
    ais AND (
        prediction OR predicting OR forecast OR forecasting
    ) AND (
        vessel OR ship OR maritime
    ) AND (
        destination OR availability OR supply OR trajectory OR logistics
    )
) AND PUBYEAR > 2014 AND (
    LIMIT-TO ( DOCTYPE , "cp" ) OR
    LIMIT-TO ( DOCTYPE , "ar" ) OR
    LIMIT-TO ( DOCTYPE , "re" ) OR
    LIMIT-TO ( DOCTYPE , "ch" ) OR
    LIMIT-TO ( DOCTYPE , "Undefined" )
) AND ( EXCLUDE ( SUBJAREA , "MEDI" ) )
\end{lstlisting}

\subsection{Results}

Using the aforementioned (\cref{sec:lit_review}) search libraries, the defined search query returned a total of \textbf{109} papers. After removing the overlapping results from the different search engines there were a total of \textbf{89} unique papers that formed the foundation of the literature review.

\todo{update numbers}
First, the papers were evaluated based on the level of relevance as defined in \cref{sec:lit_review}. Out of the \textbf{89} papers, \textbf{49} fell within \textit{category 0}, \textbf{32} within \textit{category 1}, and \textbf{6} within \textit{category 2}.

The large number of irrelevant papers resulted from the broadness of the query that was designed to find results in multiple topic areas. Furthermore, there were some papers that were medical in nature but not labeled correctly in \textit{Scopus} and were returned as the term \acrshort{ais} is also an acronym of \textit{Arterial Ischemic Stroke}. Some papers were also not publicly available but were not excluded by the search, while other papers were deemed irrelevant as they concerned topics such as mapping fishing areas in a specific region, power and performance predictions using \acrshort{ais} data, or high-level discussions of potential applications of \acrshort{ais} data analysis.

The large number of papers within \textit{category 1} further confirms the general trend of \acrshort{ais} -based predictions as the primary goal of most of the resulting papers were to predict future positions of vessels within shorter time intervals for the purpose of either safety and navigation or anomaly detection. None of the papers within \textit{category 1} seemed applicable to predict vessel destination ports at a global scale, however, for reproducibility, all papers with the relevancy of \textit{1} are listed in \todo{figure out how to add big tables of all the papers}.

% most relevant papers collected from literature review
\begin{table}[htbp]
    \centering
    \begin{tabular}{p{0.6in} p{0.8in} p{0.75in} p{0.6in} p{0.4in} p{0.8in} p{0.7in}}
    \hline
    \bfseries{Paper} & \bfseries{Goal} & \bfseries{Pred.\ method} & \bfseries{Geo-extent} & \bfseries{Time frame} & \bfseries{Validation} & \bfseries{Metrics} \\ \hline
        \cite{Karatas2020TrajectoryData} & arrival port, arrival time, and next position prediction based on trajectories & LSTM RNN, DBSCAN, K-NN & large, tested regionally & large & 10-fold cross validation & accuracy, f1-score, precision, recall \\ \hline
        \cite{Zhang2020AISApproach} & Destination prediction based on trajectories and port frequencies & Random forest, DBSCAN & global & any & 5-folder cross validation, comparison with other models & port accuracy, city accuracy, MAE, mean distance error \\ \hline
        \cite{Rosca2018GrandRoutes} & predicting arrival times and destinations of vessels & nearest neighbor search on trajectories & large, tested regionally & large & hyperparam. selection by genetic algorithm, train / test data split & general accuracy \\ \hline
        \cite{Bachar2018GrandDestination} & predicting arrival times and destinations of vessels & venilia based on markov predictive models & large, tested regionally & large & train / test data split (details lacking) & mean distance error for ETA, general accuracy for destination \\ \hline
        \cite{Dobrkovic2018MaritimeData} & longer term predictions of vessel arrival times & genetic algorithm clustering & large, tested regionally & large & case study testing, parameter testing, and simulation & general accuracy, extraction quality, execution time \\ \hline
    \end{tabular}
\caption{Papers collected from literature review with relevant geographical and time limitations}\label{tab:most_relevant_papers}
\end{table}

The remaining \textbf{}5 papers (listed in \cref{tab:most_relevant_papers}) were deemed relevant enough for further analysis in regards to the proposed research questions listed in \cref{sec:research_questions}. In addition,~\cite{lechtenberg2019}, which was discovered during the process of testing queries, was also included in the analysis as it seemed highly relevant toward availability forecasting but did not appear when using the two search engines in the final review.

\paragraphheader{RQ 1a: What prediction methods can be used to predict vessel destinations?}

Firstly, there was a large number of prediction methods that fell within \textit{category 1} that were not directly applicable for destination predictions on a global scale but were applied to smaller-scale positional predictions. These papers usually proposed some manner of clustering algorithm like \acrshort{dbscan} to classify trajectories and patterns with some form of point-to-point search along trajectories using either a form of \acrfull{rnn}, \acrfull{svm}, or \acrfull{knn} search.

A more relevant approach was proposed in~\cite{Zhang2020AISApproach} that used a \acrfull{rf} -based trajectory similarity measurement combined with frequencies of port visits to predict traveling vessels' next destination port. The \acrshort{dbscan} algorithm was used to define trajectories by identifying clusters of vessels around port coordinates. These positions were classified as ``port-stay-points'' so that every position between two stay points was part of a vessel trajectory. In the paper, the proposed \acrshort{rf}-based approach was compared with several other \acrshort{ml} and non-\acrshort{ml} methods of predicting destination ports in a similar fashion based on trajectory similarity. Using the proposed \acrshort{rf}-based approach, they achieved a ``port accuracy'' of \textit{66.57\%}, and a ``city accuracy'' of \textit{81.65\%}.

\cite{lechtenberg2019} used an ensemble approach to forecast vessel supply for the dry bulk cargo industry in predefined regions. This implicitly involved predicting port destinations in the form of voyage patterns and extrapolating port availability into regional availability. They mainly used the Markov Decision Process to predict the next destination ports together with XGBoost to predict \acrshort{eta} and anchor time before leaving the current region. The paper claims a \textit{98\%} accuracy for regional availability, however, the accuracy for port prediction was not disclosed as well as the size and extent of the predefined regions that presumably would have a massive impact of accuracy as the larger the regions are, the easier it is to predict the next region. This is especially true for smaller vessels that may rarely or never leave a large enough region.

\cite{Rosca2018GrandRoutes} and~\cite{Bachar2018GrandDestination} were both published as part of \textit{``The DEBS Grand Challenge 2018''} were the challenge involved predicting future destinations of vessels given a dataset containing historical \acrshort{ais} data within the Mediterranean sea.~\cite{Rosca2018GrandRoutes} proposes the nearest neighbor search on coordinates within trajectories where similar points are defined using both distance, speed, and heading. The longest similar sequence, or the longest predicted segment, is used as a basis for querying a vessel's arrival port.~\cite{Bachar2018GrandDestination} developed a tool called ``Venilia'' that uses different \acrshort{ml} methods to predict vessels' destinations, mainly, a Markov predictive model. Using their model, they were able to correctly classify 50\% of the events from the dataset.

\cite{Karatas2020TrajectoryData} proposes and compares a number of different \acrshort{ml}-based trajectory similarity approaches including methods proposed in entries to the \textit{``The DEBS Grand Challenge 2018''}. They also used a similar dataset contained with the Mediterranean sea spanning a few months in history. A number of prediction methods trained and evaluated using geographical and navigational parts of historical \acrshort{ais} data. They were further tested using a grid-based mapping of configurable size and resolution. In general, they found that a \acrshort{rf}-based model achieved the best results with accuracy for arrival ports around 86\%. They also found that a \acrfull{lstm} architecture performed well predicting next positions in a trajectory on a smaller scale.

\cite{Dobrkovic2018MaritimeData} proposes a genetic algorithm that is trained to cluster waypoints, and use them to establish a directed graph of sea lanes. They discuss the differences and limitations of other clustering algorithms such as \acrshort{dbscan} which is widely used for trajectory-related clustering. The proposed genetic algorithm can be applied in a similar fashion to the method described \cite{pallotta} by clustering waypoints, detecting sea lanes, and using them to predict vessels' future destination ports. The genetic algorithm is supposedly more suited for clustering in busy areas where individual vessel trajectories are hard to distinguish from others. The suggested approach shows promise toward prediction, however, the main focus of the study is to detect sea lanes while handling missing or varying data. Therefore, the complete implications in regards to destination prediction are unknown on a global scale.

\paragraphheader{RQ 1b: What information can be used to predict vessel destinations?}

From the research within \textit{category 1}, historical \acrshort{ais} data was almost exclusively the only source for predictions. Furthermore, for the most part, only purely spatial attributes of the \gls{aivdm} protocol were used. However, when compared to \textit{category 2}, papers in \textit{category 1} were more frequently using navigational information such as \acrshort{cog}, \acrshort{sog}, and headings as these features proved crucial to detect collision situations and anomalies in voyage patterns.

All papers listed in \textit{category 2} relied on historical \acrshort{ais} data and only considered the geographical coordinates clustered as trajectories for destination predictions. They all also relied on a varied collection of port data as this is necessary to predict destination ports. Since \cite{Rosca2018GrandRoutes} and \cite{Bachar2018GrandDestination} were both published in response to \textit{``The DEBS Grand Challenge 2018''} they relied on the same input data that included \acrshort{ais} and port data from a single region covering the Mediterranean sea. On the other hand, \cite{Zhang2020AISApproach} had access to 141 million global historical \acrshort{ais} records from 2011 to 2017 and a global port database consisting of over 10 000 ports.

\paragraphheader{RQ 1c: How extensive are existing prediction methods?}

As mentioned frequently in this section, all papers from \textit{category 1} were, in some manner, limited by either geographical extent or time intervals as global destination prediction was not the focus of these papers. There were slight variances in these limitations ranging from time limitations of minutes to hours with some papers considering predictions up to one day. However, since a voyage can last longer than one month in many cases, positional predictions accurate up to one day in the future do not seem applicable to this thesis' problem area.

The papers within \textit{category 1}, were less limited by time and geographical extent. For instance, \cite{Zhang2020AISApproach} and \cite{lechtenberg2019} both were completely unrestricted by geographical extent or time frames, however, the first did require a current traveling trajectory to predict the next destination port but was capable of making predictions for any voyage no matter the length or duration.

However, the solutions proposed from the remaining papers listed in \cref{tab:most_relevant_papers} were only applied, or tested, regionally.~\cite{Rosca2018GrandRoutes} and~\cite{Bachar2018GrandDestination} proposed solutions to the same challenge using the same dataset containing data within the Mediterranean sea, however, it seems as both approaches are unrestricted in terms of time limitations.~\cite{Karatas2020TrajectoryData} also proposed an apparent general destination prediction method, however, they were also limited by a dataset only covering the Mediterranean sea and only containing records spanning a couple of months. Additionally, ~\cite{Dobrkovic2018MaritimeData}'s proposed solution to long-term predictions was also only validated on limited regions which were two Dutch provinces.

In conclusion, even papers that set out for long-term destination predictions are mostly all focused on a particular area or region, and although the proposed solutions seem promising, they do not fill the global requirements of this thesis' goals. From the related work, it is apparent that only the solutions proposed in~\cite{lechtenberg2019} and~\cite{Zhang2020AISApproach} are globally extensive.

\paragraphheader{RQ 1d: How can the validity of the prediction methods be ensured?}

From \textit{category 1}, there was a multitude of different validation approaches that were relevant to smaller-scale predictions. In these methods, the most prevalent metrics used were distance-based error rates as the positional accuracy of the models is important for topics such as collision detection. Other standard \acrshort{ml} related validity metrics also occurred such as \acrfull{rmse}, \acrfull{mae}, F1-score, and general prediction, or classification, accuracy.

\cite{lechtenberg2019} does not disclose the evaluation process in much detail, however, it is mentioned that there is a standard division of 90\% training data and 10\% test data. It mentions using both \acrshort{mae} and \acrshort{rmse} as metrics measuring the quality of their approach which are both frequently used metrics \acrshort{ml}.

\cite{Zhang2020AISApproach} describes a more thorough evaluation process using 5-folder cross-validation which is a well-established process to ensure a model's accuracy is reliable and not a result of overfitting. The proposed model was compared under similar conditions with several other trajectory similarity measurements such as the aforementioned \acrshort{sspd} algorithm but also a few \acrshort{ml}-based approaches. The metrics used to indicate validity were mainly \acrfull{apde} based on the distance from the predicted port to the actual arrival port, port accuracy which was also extrapolated to city accuracy.

\cite{Karatas2020TrajectoryData} employed a 10-folder cross-validation process to validate their approach which was compared to a number of different \acrshort{ml}-based prediction models. General prediction accuracy was the main validity metric in addition to other standard metrics, namely F1-score, precision, and recall.

Neither~\cite{Bachar2018GrandDestination} nor~\cite{Rosca2018GrandRoutes} does not describe an evaluation process in high detail. However, in \cite{Rosca2018GrandRoutes}, there are mentions of testing two different data structures and their impacts on runtime performance as well as a general best score of 0.8249 for predicting arrival ports and a mean error rate for \acrshort{eta} predictions.

Finally,~\cite{Dobrkovic2018MaritimeData} is harder to compare to as destination predictions were not directly applied and tested in the study. In regards to route construction and pattern extraction, they ran tests in a simulated environment in order to establish the accuracy of their genetic algorithm that tries to establish sea lanes and routes. They reported an accuracy of 87.5\% in one simulation case and a lower score of 75\% in a scenario including missing parts in the input data which the algorithm is trained to handle.

\paragraphheader{RQ 1: How can \acrshort{ais} data combined with specific vessel details be applied to predict future destinations of maritime vessels?}

As discussed throughout this section, the existing literature within the thesis' topic area seems to exhibit a few trends. First, the majority of the research discovered mostly focused on short-term navigational predictions of vessels, usually for the purpose of collision avoidance, anomaly detection, or safety and management within ports or smaller regions. Second, there is a very limited amount of research that focuses on longer time intervals or geographical extent, and these studies are also quite limited in the sense that they rarely consider global port destination predictions, but rather port destinations within a given region. Lastly, there was no research found that considered much more than purely geographical attributes of historical \acrshort{ais} data. When only considering geographical travel patterns of vessels, it is implicitly assumed that all vessels behave in a similar fashion which, as shown in \cref{fig:segment_map}, is not the case. Since the existing literature is lacking general global destination predictions as well methods that consider specific details for individual vessels, it can be concluded that the existing literature does not provide enough insight into \texttt{RQ1} as a whole.


\begin{sidewaystable}
    \centering
    {\small
    \begin{tabular}{|l|l|l|l|l|l|l|}
    \hline
        Paper & Motivation & Prediction method & Geographical scale & Time scale & Validation method & Validation metrics \\ \hline
        \cite{Alizadeh2020PredictionTrajectory} & trajectory similarity measurement for prediction of positions at time intervals & "spatial distance, bi-directional distance, speed distance" & "region (strait of georgia, USA)" & "10, 20, 30 minutes" & "sorenson similarity index (SSI), case study in region" & accuracy \\ \hline
        \cite{Alizadeh2021VesselData} & Vessel trajectory prediction for collision avoidance & "LSTM (RNN) with trajectory distance similarity measurements (TSSP, PSSP, TSSPL)" & "Strait of Georgia, USA" & Short term (10 - 40 mins) & 1 to 8 division of training and validation set & haversine distance accuracy (0.8 km to 3.5 km from 10-40 mins) \\ \hline
        \cite{Borkowski2017TheFusion} & data fusion prediction for collision avoidance integrated in navigation system & "ANN, data fusion, GRNN" & small (collision avoidance) & small & integrated and tested in real naviagtional system & RMSE \\ \hline
        \cite{Brandt2017MovingPrediction} & short time predictions of moving objects & "moving object data stream mangement systems, kNN " & "small, region in US" & small (10 minutes) & test cases & not explained \\ \hline
        \cite{Burger2020DiscretePrediction} & trajectory predictions for filling in gaps in AIS data & "DKF (discrete kalman filters), LRM" & small & small & single cases analysis on a vessel comparing two models & MED (mean euclidea distance) \\ \hline
        \cite{Chen2020ThePrediction} & cluster reconstruction not requiring training phase for short time frames & NPC clustering finding best possible next points & small & small & extensive comparisons with other methods & "accuracy, distance error" \\ \hline
        \cite{Dalsnes2018ThePrediction} & collision detection for autonomous vessels & NCDM & small (collision detection) & small (collision detection) & 90/10 training validation sets & RMSE \\ \hline
        \cite{Dijt2020TrajectoryShips} & collision avoidance for autonomous ships & sequence to sequence neural network & small (collision avoidance) & small (collision avoidance) & "90/10 data split of six hours trajectories, cross folder validation" & "absolute trajectory error, RMSE, MAE" \\ \hline
        \cite{DIng2020ALSTM} & longer time and multidimensional trajectory predictions & LSTM & small & 5-20 minutes & "training, validation, test set (8:1:1)" & MSE \\ \hline
        \cite{Forti2020PredictionNetworks} & sequence-to-sequence RNN approach & RNN with LSTM encoder-decoder architecture & region & small & 5-fold cross validation & RMSE \\ \hline
        \cite{Guo2018TrajectoryChain} & trajectory predictions MDTN (mobile delay tolerant network) & k-order multivariate markov chain & region (grid based) & small & "simulation, experiments" & accuracy \\ \hline
        \cite{Hexeberg2017AIS-basedPrediction} & collision detection & single neighbor search (SPNS) & region (trondheim) & small (10 minutes) & "training, validation sets, manually selected scenarios, validate with real trajectory" & RMSE \\ \hline
        \cite{Jin2020MaritimeNetwork} & longer range predictions for security motivations & "RNN, LSTM" & regional/small & small & model simulation & "distance accuracy over time, MAE, SSE" \\ \hline
        \cite{Kim2018PreprocessingArea} & predictions for Vessel Traffic Service (VTS) & NN & small & small & case study on region & speed and distance error \\ \hline
        \cite{Li2018ShipMining} & ceaner data extraction and mining for predictions & RBF neural network model & small (tested on river in china) & small-medium (hours) & simulation/case study on river in china & trajectory difference from real to simulated \\ \hline
        \cite{Li2019Long-termData} & longer term predictions for collision avoidance & "LSTM, longest comomon subsequence(LCS) algorithm, DBSCAN clustering" & small / collision avoidance & small / collision avoidance (15 min) & "applied to 4 regions, case studies" & distance error \\ \hline
        \cite{Lian2019ResearchAlgorithm} & "investigating particle filtering, near prediction, least squares estimation approach to predictions for smaller scale predictions" & "linear prediction, least squares, particle filtering" & small / collision avoidance & small / collision avoidance & "simulation, 9 hours of data" & "distance error, speed error" \\ \hline
        \cite{Liu2019VesselACDE-SVR} & trajectory prediction that also handles real time at sea & "SVR, ACDE, RNN" & small / collision avoidance & small / collision avoidance & training/validation sets & distance error \\ \hline
        \cite{Liu2020PredictingLearning} & "predicting trajectories for ship management, interpolating method for filling in missing AIS in trajectory" & "LSS-VM (least-squares support-vector machines) for predictions, cubic spline function to regulate trajectories via interpolation" & "independent of regions, but on a small scale (predicting distance not arrival ports)" & small (predicting next positions in trajectory) & four random trajectories selected for predictions & accuracy in distance/meters from actual trajectory \\ \hline
        \cite{Mao2018AnMining} & database for trajectory prediciton and mining & "interpolating trajectory reconstruction, are of interest bsed grid search, ELM and SLFN for predictions" & region based & medium (20 - 40 minutes) & original vs predicted trajectory & distance error \\ \hline
        \cite{Murray2018AOperations} & collision detection for autonomous vessels & Single point neighboir search method & small (collision detection) & 5-30 minutes & 90/10 training validation sets & RMSE \\ \hline
        \cite{Murray2019AnVessels} & collision avoidance & "gaussian mixture modelling, principle component analysis" & small / collision avoidance & small / collision avoidance & running 100 times randomly selecting points & distance error \\ \hline
        \cite{Murray2020AData} & trajectory predictions for early warnings and safety & "GMM clustering (gaussian mixture model), novel dual autoencoder" & region (tromsø) & 30 minutes (1 year or historical AIS) & distributed accuracy over time in the future (predicted positions vs actual positions) & accuracy at time intervals \\ \hline
        \cite{Rong2019ShipModel} & modelling uncertainty of trajectory predictions & "Bayesion model, Gaussian Process" & small / collision avoidance & "small / collision avoidance (10, 20, 30)" & "case study in region, training / validation data" & "accuracy, distance error" \\ \hline
        \cite{Suo2020ANetwork} & trajectory predictions for early warnings and safety & "GRU (gate recurrent unit), DBSCAN, comp. with LSTM" & tested on single port in china & "small, minutes to hour" & "training, validation, test set (not defined how much)" & accuracy \\ \hline
        \cite{Tafa2019AutomaticPrediction} & synthetic route representation and predictions & "DBSCAN, route similarity probability model" & east china sea region & 10-80 minutes & simulation & accuracy \\ \hline
        \cite{Tang2019ANetwork} & collision avoidance for automonous ships & LSTM & region in china & uses 10 min of data to predict 20 next minutes & training/validation sets & "MAE, MSE" \\ \hline
        \cite{Uney2019DataModels} & forecasting trajectories from historical and streaming trajectories & directed grid based bayesion model / gaussian mixture forecast density & tested on region (15x15 grid) & any (tested with 2 months of data) & real life case study in region & not explained \\ \hline
        \cite{Virjonen2018ShipMethod} & predictions in area in finnland that has to be several hours ahead in time & k-nearest neighbours & medium (region of finnland test case) & "medium, several hours" & nested leave-one-out-cross-validation (LOOCV) & distance accuracy \\ \hline
        \cite{Wang2020VesselGRU} & predicting vessel berthing trajectory for safety and collision avoidance & "Bi-GRU (tensorflow, keras)" & single port in china & small (minutes) & "training, validation set (not defined ratio), compared to other models" & MSE \\ \hline
        \cite{Xiao2020BigTechniques} & "for collision avoidance, better quering, more effective predictions" & "knowledge based particle filtering (PF), MLNN" & "smaller, limited to collision avoidance" & 3-10 minutes & testing different scnarios i.e. case studies & "sog, coc, and distance error" \\ \hline
        \cite{You2020ST-Seq2Seq:Prediction} & sequence-to-sequence RNN approach & "seq2se1 GRU, RNN, encoder/decoder" & "small, limited to 10m trajectories" & 10 minutes & "analysis in region (few rivers in china), training/validation set" & "AdaGrad, RMSProp" \\ \hline
        \cite{Zheng2020HeterogenousModeling} & combining multiple datasources like GPS and ARPA with AIS to improve predictions for safety & LSTM (on different data and a fusion component to merge the predictions) & small & small & "training, validation (1:10), and compare to other model" & MSE \\ \hline
        \cite{Zhou2019ShipNetwork} & collision avoidance in busy areas & back propagation nerual network & region (area in china) & small & training/validation 70/30 & RMSE \\ \hline
    \end{tabular}
    }
\end{sidewaystable}
