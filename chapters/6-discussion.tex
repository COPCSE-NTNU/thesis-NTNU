\chapter{Discussion}

\section{Summary}

\section{Data architecture}

\begin{itemize}
    \item Scalability, availability. PostGreSQL DB structure advantages/disadvantages
    \item Trajectory definitions, full voyages vs port-to-port voyages
\end{itemize}

\subsection{Trajectory sampling}

\begin{itemize}
    \item Benefits of sampling.
    \item Different approaches to sampling.
\end{itemize}

\section{ML training structure}

\begin{itemize}
    \item MSTD
    \item Dataset imbalance
\end{itemize}

\section{ML-based destination prediction}

\begin{itemize}
    \item Performance
    \item Weaknesses/limitations
    \item Dataset vs real life
\end{itemize}

\section{Real-life applications}

\begin{itemize}
    \item What is the value of this solution
    \item What can it be used for
    \item Is this something that contributes to the industry as well as academia
    \item Value for MO. future applications and integrations in MO's product (teasers)
\end{itemize}

\section{Limitations and future work}

\begin{itemize}
    \item Voyage definition - anchoring, bunkering
    \item AIS navigational status is manual input, can we better claim that a vessel has arrived/departed a port?
    \item MSTD - trajectory similarity measurement. Zhang et al. achieved better results using RF than SSPD.
    \item Improving the model by adding more features such as weather, seasons, ballast/laden, \ldots more?
    \item Implementing at large scale providing on-demand predictions and availability
\end{itemize}
