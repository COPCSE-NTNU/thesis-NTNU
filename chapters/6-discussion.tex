\chapter{Discussion}

\section{Summary}

\begin{itemize}
    \item Brief summary of the entire process
\end{itemize}

\section{Dataset and problem formulation}

\subsection{Vessel voyage definition}

\subsection{Geographical trajectory abstraction and MSTD}
\begin{itemize}
    \item Is MSTD an appropriate abstraction of trajectories.
    \item Are there other ways to combine trajectory predictions with additional voyage and vessel data?
    \begin{itemize}
        \item MSTD with more filters?
        \item Different trajectory similarity measurements?
        \item ML-based trajectory similarity normalized with port frequency considering segments?
    \end{itemize}
\end{itemize}

\subsection{Data preparation and dataset imbalance}
\begin{itemize}
    \item How was encoding appropriate?
    \item Was sampling appropriate, how did it affect results?
    \item Are there different sampling methods that could have been more appropriate?
\end{itemize}

\subsection{Data storage and architecture}
\begin{itemize}
    \item Scalability, availability. PostGreSQL DB structure advantages/disadvantages
\end{itemize}

\section{ML-based destination prediction}
\begin{itemize}
    \item Why did we get the described performance.
    \item Weaknesses/limitations
    \item Dataset vs real life
\end{itemize}

\section{Research questions and hypothesis}

\begin{itemize}
    \item How did I answer the research questions?
    \item Any additional hypothesis?
    \item How was what I did different from related works?
\end{itemize}

\subsection{Feature importances}

\subsection{Segment predictability}

\begin{itemize}
    \item Why are some segments easier to predict
    \item Why are some sub-segments easier to predict
\end{itemize}

\section{Real-life applications}

\begin{itemize}
    \item What is the value of this solution
    \item What can it be used for
    \item Response from external actors
    \item Is this something that contributes to the industry as well as academia
    \item Value for MO\@. Future applications and integrations in MO's product (teasers)
\end{itemize}

\section{Limitations and future work}

\begin{itemize}
    \item Voyage definition - anchoring, bunkering
\begin{itemize}
    \item AIS navigational status is manual input, can we better claim that a vessel has arrived/departed a port?
    \item clustering vs. transitions
\end{itemize}
    \item MSTD - trajectory similarity measurement. Zhang et al. achieved better results using RF than SSPD.
    \item Improving the model by adding more features such as weather, seasons, ballast/laden, \ldots more?
    \item Implementing at large scale providing on-demand predictions and availability
\begin{itemize}
    \item Implement a real-time updated voyage database
    \item Fetch every vessels current trajectory and estimate MSTD
    \item Pass all vessel data to ML model which, very quickly, predicts destination ports.
    \item Use a routing concept to get ETA (mesh-based routing, ais-based routing)
\end{itemize}
\end{itemize}

\subsection{Vessel availability}
\begin{itemize}
    \item Further implementation with MO's infrastructure.
    \item Give me all vessels that will be in port A at X time.
\end{itemize}
