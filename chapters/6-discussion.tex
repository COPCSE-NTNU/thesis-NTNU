\chapter{Discussion}

\section{Summary}

\begin{itemize}
    \item Brief summary of the entire process
\end{itemize}

\section{Dataset and problem formulation}

\subsection{Vessel voyage definition}

\subsection{Geographical trajectory abstraction and MSTD}

\subsection{Data preparation and dataset imbalance}

\subsection{Data storage and architecture}
\begin{itemize}
    \item Scalability, availability. PostGreSQL DB structure advantages/disadvantages
\end{itemize}

\section{ML-based destination prediction}
\begin{itemize}
    \item Why did we get the described performance.
    \item Weaknesses/limitations
    \item Dataset vs real life
\end{itemize}

\section{Real-life applications}

\begin{itemize}
    \item What is the value of this solution
    \item What can it be used for
    \item Response from external actors
    \item Is this something that contributes to the industry as well as academia
    \item Value for MO\@. Future applications and integrations in MO's product (teasers)
\end{itemize}

\section{Limitations and future work}

\begin{itemize}
    \item Voyage definition - anchoring, bunkering
\begin{itemize}
    \item AIS navigational status is manual input, can we better claim that a vessel has arrived/departed a port?
    \item clustering vs. transitions
\end{itemize}
    \item MSTD - trajectory similarity measurement. Zhang et al. achieved better results using RF than SSPD.
    \item Improving the model by adding more features such as weather, seasons, ballast/laden, \ldots more?
    \item Implementing at large scale providing on-demand predictions and availability
\begin{itemize}
    \item Implement a real-time updated voyage database
    \item Fetch every vessels current trajectory and estimate MSTD
    \item Pass all vessel data to ML model which, very quickly, predicts destination ports.
    \item Use a routing concept to get ETA (mesh-based routing, ais-based routing)
\end{itemize}
\end{itemize}
