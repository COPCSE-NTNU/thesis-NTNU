\chapter{Discussion}
\label{chap:discussion}

In this chapter, a brief summary of the thesis is provided followed by discussions regarding the proposed solution, the field of study, possible applications and the validity of the approach in terms of both academic and commercial values. Finally, limitations of the thesis and proposed future work is presented and discussed.

\section{Summary}

This thesis has investigated the topic of destination prediction for vessels in the shipping industry by using historical \acrfull{ais} data and additional vessel information. \acrshort{ais} is a globally adopted tracking system that transmits all commercial vessels' geographical and navigational information similar to that of GPS\@. The thesis objective was to develop a methodology for predicting traveling vessels' future destination ports on a global scale unrestricted by specific vessel types, geographical regions, or time intervals using historical \acrshort{ais} data. The thesis was developed in collaboration with a technological maritime start-up company called \acrfull{mo} who provided the data foundation used throughout the thesis. The thesis aimed to answer two primary research questions:

\begin{enumerate}
    \item How can \acrshort{ais} data combined with specific vessel details be applied to predict future destinations of maritime vessels?
    \item What is the impact of vessel segmentation on prediction methods, or vessels' general predictability?
\end{enumerate}

Related work within the area of vessel destination or trajectory prediction was investigated to determine to what extent existing literature had already answered the research questions. It was found that the majority of related work was motivated by collision avoidance for safety reasons, anomaly detection to detect vessels deviating from established shipping lanes, automated collision avoidance systems to be installed on autonomous vessels, or short term trajectory predictions to aid in port management and scheduling. Existing works motivated by these factors did not consider the future destination port of the traveling vessels, but rather the vessels' specific positions or future trajectories in a short time interval ranging from minutes to a few hours. The few related studies that considered future destinations of vessels were almost all limited, or exclusively tested on a specific region or area such as the Mediterranean sea. One study was found that considered a general port destination prediction approach, however, this study, presented in \cite{Zhang2020AISApproach}, exclusively considered the geographical information provided by the \acrshort{ais} standard, and did not consider additional vessel information such as the type, size, or capacity of vessels. Thus, since the research questions were close to unanswered by the existing literature, the thesis goal was refined to developing a global and general vessel destination prediction method that is capable of considering more than purely spatial voyage information such as vessel segments and sub-segments as provided by the collaborative company \acrfull{mo}.

In order to use spatial voyage trajectories derived from \acrshort{ais} data in the final prediction method, the thesis formulated a voyage definition that determined what conditions had to be true to consider a vessel to be arrived at a specific port. Based on related works, a clustering-based approach was initially evaluated that detected ``clusters'' of \acrshort{ais} records close to shipping ports using the \acrfull{dbscan} algorithm. When positional records were transmitted in a high enough density close to a port, the vessel was considered arrived. However, this approach was later abandoned in favor of a voyage definition that considered the vessels themselves expressing an arrived state via the navigational status ``moored'' in the \acrshort{ais} data. The latter approach was favored as it ignored intermediate port visits during a voyage such as when vessels stopping to refuel at bunker ports, it therefore produced more valuable voyages that ends where the vessel unloads cargo and considers itself to be moored.

Using a standardized voyage definition, historical \acrshort{ais} data ranging from December 2019 to March 2021 provided by \acrshort{mo} was constructed into \textbf{1.7} million voyages and trajectories defined as positional records transmitted between subsequent departures and arrivals. These voyages formed the initial training data to be used to train a \acrfull{ml} model to predict voyages' arrival ports. In order to consider specific vessel information, voyage information, and spatial trajectories, a method of structuring spatial trajectories as categorical and numerical values was proposed. In this approach every historical trajectory was compared with every other trajectory outgoing from the same departure port from vessels of the same type in order to find the most similar historical trajectory. The \acrfull{sspd} algorithm was used to determine trajectory similarity. The trajectories had been simplified prior to this comparison by only using one point at every six hour interval in each trajectory in order to make comparisons easier. Furthermore, in the process of trajectory comparisons, each voyage was divided into at most four incomplete versions of the same trajectory to emulate realistic voyages not yet reached their final destination. The most similar historical trajectory's destination port (\acrshort{mstd}), the value indicating the similarity value of the trajectories, and the trajectory length became the categorical and numerical values representing a voyage's spatial trajectory. The final training set only contained categorical and numerical values, consisted of \textbf{4.3} million incomplete voyages, and formed the final training dataset used to train a \acrshort{ml} model.

An \acrfull{xgb} \acrshort{ml} model was then configured and trained to predict a voyage's arrival port by considering the vessel's segmentation, or type, departure port, \acrshort{mstd}, \acrshort{mstd} similarity, and trajectory length, or duration. The training process used \textit{80\%} and \textit{20\%} as training and evaluation data respectively and achieved an accuracy score of \textbf{72\%}, and an F1-score of \textbf{0.734} validated by additional metrics and cross-folder validation.

\section{Dataset and problem formulation}

\subsection{Vessel voyage definition}

\subsection{Geographical trajectory abstraction and MSTD}
\begin{itemize}
    \item Is MSTD an appropriate abstraction of trajectories.
    \item Are there other ways to combine trajectory predictions with additional voyage and vessel data?
    \begin{itemize}
        \item MSTD with more filters?
        \item Different trajectory similarity measurements?
        \item ML-based trajectory similarity normalized with port frequency considering segments?
    \end{itemize}
\end{itemize}

\subsection{Data preparation and dataset imbalance}
\begin{itemize}
    \item How was encoding appropriate?
    \item Was sampling appropriate, how did it affect results?
    \item Are there different sampling methods that could have been more appropriate?
\end{itemize}

\subsection{Data storage and architecture}
\begin{itemize}
    \item Scalability, availability. PostGreSQL DB structure advantages/disadvantages
\end{itemize}

\section{ML-based destination prediction}
\begin{itemize}
    \item Why did we get the described performance.
    \item Weaknesses/limitations
    \item Dataset vs real life
\end{itemize}

\section{Research questions and hypothesis}

\begin{itemize}
    \item How did I answer the research questions?
    \item Any additional hypothesis?
    \item How was what I did different from related works?
\end{itemize}

\subsection{Feature importances}

\begin{itemize}
    \item MSTD
    \item Segmentation
    \item weights
    \item good setup to add new features as it is easy to see impact
\end{itemize}


\subsection{Segment predictability}

\begin{itemize}
    \item Why are some segments easier to predict
    \item Why are some sub-segments easier to predict
\end{itemize}

\section{Real-life applications}

\begin{itemize}
    \item What is the value of this solution
    \item What can it be used for
    \item Response from external actors
    \item Is this something that contributes to the industry as well as academia
    \item Value for MO\@. Future applications and integrations in MO's product (teasers)
\end{itemize}

\section{Limitations and future work}

\begin{itemize}
    \item Further investigation of imbalanced datasets, over optimistic evaluation, and oversampling. (https://ieeexplore.ieee.org/abstract/document/8492368)
    \item What are appropriate sampling methods?
    \item The model is not overfitted, but is it overoptimistic?
\end{itemize}

\begin{itemize}
    \item Voyage definition - anchoring, bunkering
\begin{itemize}
    \item AIS navigational status is manual input, can we better claim that a vessel has arrived/departed a port?
    \item clustering vs. transitions
\end{itemize}
    \item MSTD - trajectory similarity measurement. Zhang et al. achieved better results using RF than SSPD.
    \item Improving the model by adding more features such as weather, seasons, ballast/laden, \ldots more?
    \item Implementing at large scale providing on-demand predictions and availability
\begin{itemize}
    \item Implement a real-time updated voyage database
    \item Fetch every vessels current trajectory and estimate MSTD
    \item Pass all vessel data to ML model which, very quickly, predicts destination ports.
    \item Use a routing concept to get ETA (mesh-based routing, ais-based routing)
\end{itemize}
\end{itemize}

\subsection{Vessel availability}
\begin{itemize}
    \item Further implementation with MO's infrastructure.
    \item Give me all vessels that will be in port A at X time.
\end{itemize}
