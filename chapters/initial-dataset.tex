\chapter{Initial dataset}

\section{Vessel positions history}

\section{Vessel segments}
\label{sec:vessel_segments}

(copied from APW)

...

At a different data location, \acrshort{mo} stores the segmentation for any vessel ever picked up by the \acrshort{ais} transmitters or collected from other sources. These values include the vessel’s segment and sub-segment which is a categorization and sub-categorization that defines the type of the vessel. These values are based on data such as the dimensions of the vessel as well as data collected from external sources and even MO’s users. Some vessels also have a variation within a sub-segment, however, since this only applies to some sub-segments and a relatively few amount of vessels, this was not included in this report as it would be difficult to accurately compare port frequencies as some ports are never visited by vessels with variation values.

\section{Vessel transitions}
\label{sec:vessel_transitions}

(copied from APW)

\acrfull{mo} stores vessel transitions for more than 60 000 vessels collected from \acrfull{ais} data. A vessel transition consists of values based on the navigational status and position of the vessel. The navigational status which is part of the \Gls{aivdm} protocol provides high-level navigation data about the vessel such as \textit{“Underway using engine”}, \textit{“At anchor”}, \textit{“Moored”, “Underway sailing”}, etc. As an example of a vessel transition, any vessel that exists within a given radius of any port and switches her \acrshort{ais} navigational status to “moored” is labeled as “arrived” at the closest port. When a vessel changes from “moored” to other statuses that indicate the vessel is moving, another transition is stored with the transition type of “departed” from the given port.

The values stored for each transition includes unique identifiers for the vessel such as the \acrfull{mmsi} and \acrfull{imo} numbers, the unique id of the given port, the transition type, i.e. whether the vessel arrived or departed the port, and a timestamp for when the transition took place. 

\section{Ports}
\label{sec:port_data}

(copied from APW)

...

Port ids correspond to a unique port that is part of a global collection of shipping ports consisting of over 5200 ports. The port id used in this report is a unique five letter \Gls{locode} port code where the first two letters correspond to the origin country of the port, and the latter three to a more specific location of a port. For instance, the port id \texttt{NOOSL} corresponds to Oslo port in Norway. For reference, a similar system is also used for airports throughout the world.