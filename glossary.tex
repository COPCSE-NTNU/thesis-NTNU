
% From https://www.overleaf.com/learn/latex/Glossaries

\makeglossaries % Prepare for adding glossary entries


\newglossaryentry{latex}{
        name=latex,
        description={Is a mark up language specially suited for
scientific documents}
}

\newglossaryentry{bibliography}{
    name=bibliography,
    plural=bibliographies,
    description={A list of the books referred to in a scholarly work, typically printed as an appendix}
}

\newglossaryentry{maths}{
    name=mathematics,
    description={Mathematics is what mathematicians do}
}

\newglossaryentry{aivdm}{
    name=AIVDM/AIVDO,
    description={Protocol for sentences transmitted by AIS transmitters}
}

\newglossaryentry{locode}{
    name=UN/LOCODE,
    description={Five-letter geographic coding scheme maintained by the UN. The codes are assigned to, among others, ports where the first to letters represents a country code and the remaining three represents a location}
}


% --------------------
% ----- Acronyms -----
% --------------------

\newacronym{phd}{PhD}{philosophiae doctor}
\newacronym{CoPCSE}{CoPCSE@NTNU}{Community of Practice in Computer ScienceEducation at NTNU}
\newacronym{gcd}{GCD}{Greatest Common Divisor}
\newacronym{mo}{MO}{Maritime Optima AS}
\newacronym{ais}{AIS}{Automatic Identification Systems}
\newacronym{gis}{GIS}{Geographical Information System}
\newacronym{imo}{IMO}{International Maritime Organization}
\newacronym{mmsi}{MMSI}{Maritime Mobile Service Identity}
\newacronym{ntnu}{NTNU}{Norwegian University of Technology and Science}
